%%%%%%%%%%%%%%%%%%%%%%%%%%%%%%%%%%%%%%%%%%%%%%%%%%%%%%
% Based on THU beamer theme                          %
% Author: EL HASSANI NABIL                           %
% Date: September 2024                               % 
%%%%%%%%%%%%%%%%%%%%%%%%%%%%%%%%%%%%%%%%%%%%%%%%%%%%%%
\documentclass[aspectratio=169]{beamer}
\usepackage[scaled=0.9]{helvet}
\usepackage{courier}
\usepackage[T1]{fontenc}
\usepackage{hyperref}
\usepackage{latexsym,amsmath,xcolor,multicol,booktabs,calligra}
\usepackage{graphicx,pstricks,listings,stackengine}
\usepackage{inputenx} 

\titlegraphic{\includegraphics[height=1.5cm]{pic/efrei_logo.png}}
\usetheme{Madrid}
\usecolortheme{seahorse}
\usecolortheme{seahorse}

% defs
\def\cmd#1{\texttt{\color{red}\footnotesize $\backslash$#1}}
\def\env#1{\texttt{\color{blue}\footnotesize #1}}
\definecolor{deepblue}{rgb}{0,0,0.5}
\definecolor{deepred}{RGB}{153,0,0}
\definecolor{deepgreen}{rgb}{0,0.5,0}
\definecolor{halfgray}{gray}{0.55}

\lstset{
    basicstyle=\ttfamily\small,
    keywordstyle=\bfseries\color{deepblue},
    emphstyle=\ttfamily\color{deepred},    % Custom highlighting style
    stringstyle=\color{deepgreen},
    numbers=left,
    numberstyle=\small\color{halfgray},
    rulesepcolor=\color{red!20!green!20!blue!20},
    frame=shadowbox,
}

\usepackage{UoWstyle}
%==========================================================
\begin{document}
%==========================================================
% Page de titre
\begin{frame}
    \titlepage
\end{frame}

% Plan de la présentation
\begin{frame}
    \frametitle{Plan de la Présentation}
    \scriptsize
    \tableofcontents[hideallsubsections]
\end{frame}

%==========================================================
\section{Etude de cas}
  \subsection{Strategy}
  \begin{frame}
    \textimg{0.6}{
      \textbf{Contexte}: Prenant l'exemple d'une entreprise qui développe un jeu de simulation de mare de canards très populaire, \textit{SimUDuck}. 
      Le jeu peut afficher une grande variété d'espèces de canards nageant et émettant des bruits de coin-coin. 
      Les concepteurs initiaux du système ont utilisé des techniques standard de programmation orientée objet et ont créé une superclasse 
      \textbf{Duck} dont tous les autres types de canards héritent.
    }{pic/diagrams/duck_uml.drawio.png}
      
  \end{frame}
  \begin{frame}
      \textimg{0.6}{
        \begin{itemize}
          \item Le pattern Strategy permet de définir une famille d'algorithmes et de les rendre interchangeables.
          \item Le comportement peut être changé dynamiquement sans modifier le code du client.
        \end{itemize}
      }{ pic/diagrams/strategy.png }
  \end{frame}

  \begin{frame}{Avantages du Pattern Strategy}
    \begin{itemize}
        \item \textbf{Séparation des comportements} : Permet de séparer les algorithmes du code principal en les encapsulant dans des classes distinctes.
        \item \textbf{Facilité d'extension} : Il est facile d'ajouter de nouveaux algorithmes ou stratégies sans modifier les classes existantes.
        \item \textbf{Respect du principe Open/Closed} : Les classes sont ouvertes à l'extension mais fermées à la modification, facilitant les évolutions du système.
        \item \textbf{Élimine les conditions complexes} : Remplace les multiples conditions (\texttt{if} ou \texttt{switch}) par des stratégies configurables.
        \item \textbf{Réutilisabilité} : Les stratégies peuvent être réutilisées dans différents contextes si elles sont bien conçues.
    \end{itemize}
  \end{frame}

  \begin{frame}{Inconvénients du Pattern Strategy}
    \begin{itemize}
        \item \textbf{Augmentation du nombre de classes} : Chaque stratégie est encapsulée dans sa propre classe, ce qui peut augmenter la complexité et le nombre de classes à gérer.
        \item \textbf{Complexité de maintenance} : Si un grand nombre de stratégies sont ajoutées, il peut être difficile de maintenir et de comprendre le système.
        \item \textbf{Problème de visibilité} : Les utilisateurs doivent savoir quelle stratégie utiliser et comment les configurer, ce qui peut rendre l'API plus complexe à utiliser.
        \item \textbf{Délégation de logique} : Le contexte doit souvent déléguer les appels à la stratégie appropriée, ce qui peut entraîner un couplage indirect.
    \end{itemize}
  \end{frame}

  \begin{frame}
    \frametitle{Conclusion}
    \begin{itemize}
        \item Le pattern Strategy favorise la flexibilité et permet de changer l'algorithme utilisé sans modifier la logique du contexte.
        \item Il encourage également l'adhésion au principe Open/Closed (OCP) du SOLID.
    \end{itemize}
  \end{frame}

  \subsection{Observer}
  \begin{frame}{Description de l'application Weather-O-Rama}
    \textbf{Contexte} :
    \begin{itemize}
        \item \textbf{Weather-O-Rama} est une application de suivi et d'affichage des conditions météorologiques.
        \item Elle collecte des données telles que la température, l'humidité, et la pression atmosphérique.
        \item Ces informations sont affichées sur différents panneaux.
    \end{itemize}
    
    \textbf{Composants} :
    \begin{itemize}
        \item \textbf{Station Météo} : Collecte des données climatiques (température, humidité, pression) en temps réel.
        \item \textbf{Affichages} :
        \begin{itemize}
            \item \textbf{Conditions actuelles} : Température et humidité actuelles.
            \item \textbf{Prévisions} : Tendances météorologiques futures basées sur la pression atmosphérique.
            \item \textbf{Statistiques} : Moyennes historiques des données climatiques.
        \end{itemize}
    \end{itemize}
    
    \textbf{Mécanisme de mise à jour} :
    \begin{itemize}
        \item Les affichages reçoivent automatiquement les mises à jour de la station météo.
        \item Chaque nouvelle donnée est envoyée aux panneaux d'affichage.
    \end{itemize}
    
    \textbf{Résultat} :
    \begin{itemize}
        \item Les utilisateurs peuvent surveiller les conditions en temps réel sur différents panneaux.
        \item Chaque panneau affiche des informations spécifiques pour une vue complète du climat.
    \end{itemize}
  \end{frame}

  \begin{frame}{Définition du pattern Observer}
    \only<1>{
      \begin{itemize}
        \item Le \textbf{pattern Observer} est un \textbf{design pattern comportemental}.
        \item Il définit une relation de dépendance entre objets où un objet, appelé \textbf{Sujet (Subject)}, notifie automatiquement plusieurs *\textbf{Observateurs (Observers)} de ses changements d'état.
        \item Ce pattern permet de gérer des notifications automatiques de manière efficace sans couplage fort entre les objets.
    \end{itemize}
    }
    \only<2>{
    \textimg{0.6}{
        \textbf{Le pattern Observer comprend trois composants clés} :
        \begin{itemize}
            \item \textbf{Sujet (Subject)} :
            \begin{itemize}
                \item Maintient une liste d'observateurs.
                \item Notifie les observateurs en cas de changement d'état.
            \end{itemize}
            \item \textbf{Observateurs (Observers)} :
            \begin{itemize}
                \item S'enregistrent auprès du sujet pour recevoir les mises à jour.
                \item Réagissent aux changements du sujet.
            \end{itemize}
            \item \textbf{Notifications} :
            \begin{itemize}
                \item Le sujet notifie tous les observateurs lorsqu'il y a un changement.
                \item Les observateurs mettent à jour leur état en conséquence.
            \end{itemize}
         \end{itemize}
      }{pic/diagrams/observer.png}
    }
  \end{frame}
  \begin{frame}{Avantages du pattern Observer}
    \begin{itemize}
        \item \textbf{Découplage faible} : Le sujet ne dépend pas des détails des observateurs, ce qui réduit le couplage.
        \item \textbf{Extensibilité} : Ajouter de nouveaux observateurs est facile et n'affecte pas le sujet.
        \item \textbf{Notification automatique} : Les observateurs sont automatiquement informés des changements d'état.
    \end{itemize}
  \end{frame}

  \begin{frame}{Inconvénients du pattern Observer}
    \begin{itemize}
        \item \textbf{Cascade de mises à jour} : Si le nombre d'observateurs est élevé, les notifications peuvent devenir inefficaces.
        \item \textbf{Ordre de notification} : Les observateurs sont généralement notifiés sans ordre spécifique, ce qui peut poser des problèmes dans certains cas.
    \end{itemize}
  \end{frame}

  \subsection{Decorator}
  \begin{frame}{Contexte de Starbuzz Coffee}
        \textbf{Contexte :} 
        \par En pleine expansion, Starbuzz Coffee fait face à de nouveaux défis. Leur succès fulgurant, accompagné de l'ouverture massive de nouvelles enseignes, 
        les oblige à revoir leurs infrastructures internes. En particulier, leurs systèmes de commande peinent à suivre l’évolution de leur offre de boissons, 
        qui s’est considérablement diversifiée au fil du temps. Lorsqu'ils ont débuté, ils avaient conçu leurs classes de manière simple, mais cette approche montre aujourd’hui ses limites.
  \end{frame}

  \begin{frame}{Défi de la gestion des options}
    \begin{itemize}
        \item Chaque boisson (café, thé, etc.) peut être personnalisée avec plusieurs ajouts (lait, mousse, chocolat, etc.).
        \item Si chaque combinaison possible de boisson et d'ajout était modélisée par une sous-classe, le système deviendrait rapidement ingérable.
        \item Comment permettre la personnalisation des boissons de manière flexible sans avoir à créer une classe pour chaque combinaison ?
    \end{itemize}
  \end{frame}

  \begin{frame}{Description du Pattern Decorator}
    \textimg{0.7}{
      \begin{itemize}
          \item Le \textbf{pattern Decorator} est un design pattern \textbf{structurel} qui permet d'ajouter dynamiquement des fonctionnalités à un objet sans modifier son code d'origine.
          \item Il encapsule les comportements dans des objets séparés appelés \textbf{décorateurs}, qui enrichissent un objet de base.
          \item Ce pattern suit le principe de composition \textbf{plutôt que} l'héritage pour étendre les fonctionnalités d'un objet.
          \item Chaque décorateur \textbf{enveloppe} l'objet de base et peut ajouter de nouveaux comportements ou modifier ceux existants.
          \item Les objets décorateurs et l'objet décoré partagent la même interface, ce qui permet de les utiliser de manière interchangeable.
      \end{itemize}
    }{pic/diagrams/decorator.png}
  \end{frame}

  \begin{frame}{Inconvénients du Pattern Decorator}
    \begin{itemize}
        \item \textbf{Complexité accrue} : L'utilisation excessive de décorateurs peut rendre la conception plus complexe et difficile à comprendre, surtout si de nombreux décorateurs sont empilés.
        \item \textbf{Nombre élevé de petites classes} : Chaque fonctionnalité ajoutée nécessite une nouvelle classe, ce qui peut entraîner une prolifération de petites classes dans le système.
        \item \textbf{Difficulté de débogage} : Avec plusieurs décorateurs imbriqués, il peut être difficile de suivre le flux des appels et de comprendre où une fonctionnalité spécifique est ajoutée ou modifiée.
        \item \textbf{Couplage indirect} : Les décorateurs ajoutent souvent des couches supplémentaires entre le client et l'objet réel, ce qui peut introduire un couplage indirect entre ces composants.
    \end{itemize}
  \end{frame}

  \subsection{Factory}
  \begin{frame}{Contexte de PizzaStore}
      \textbf{Contexte :} 
        Le but est de modéliser une chaîne de pizzerias où chaque franchise peut avoir ses propres types de pizzas (New York Style, Chicago Style, etc.).
  \end{frame}
    
  \begin{frame}{Description du Pattern Factory}
    \textimg{0.7}{
      \begin{itemize}
          \item Le \textbf{pattern Factory} est un \textbf{design pattern créationnel} qui fournit une interface pour créer des objets sans exposer la logique de création au client.
          \item Il permet aux sous-classes de définir quelle instance concrète doit être créée, laissant ainsi la responsabilité de l'instanciation aux sous-classes.
          \item Le pattern encapsule la logique d'instanciation dans une méthode spécialisée appelée \textbf{méthode fabrique}.
          \item Cela permet de rendre le code plus flexible et modulaire tout en facilitant l'extension du système.
      \end{itemize}
    }{pic/diagrams/factory-method.png}
  \end{frame}

  \begin{frame}{Avantages du Pattern Factory}
    \begin{itemize}
        \item \textbf{Encapsulation de la logique de création} : La méthode fabrique centralise la création des objets, ce qui rend le code client plus simple.
        \item \textbf{Respect du principe Open/Closed} : Le pattern permet d'ajouter de nouvelles sous-classes ou types de produits sans modifier le code existant.
        \item \textbf{Réduction du couplage} : Le client n'a pas besoin de connaître les classes concrètes utilisées, ce qui réduit le couplage entre les composants.
        \item \textbf{Flexibilité} : Idéal pour les systèmes nécessitant l'instanciation de produits complexes ou dépendant de plusieurs facteurs.
    \end{itemize}
  \end{frame}

  \begin{frame}{Inconvénients du Pattern Factory}
    \begin{itemize}
        \item \textbf{Complexité accrue} : Introduire des fabriques et des sous-classes peut augmenter la complexité du système.
        \item \textbf{Prolifération des classes} : Chaque nouveau produit ou sous-classe entraîne la création de nouvelles classes, ce qui peut alourdir le code.
        \item \textbf{Dépendance indirecte} : Bien que le couplage direct soit réduit, des modifications dans les sous-classes ou les produits peuvent nécessiter des changements dans la fabrique.
    \end{itemize}
  \end{frame}

  \begin{frame}{Description du Pattern Abstract Factory}
    \textimg{0.7}{
      \begin{itemize}
          \item Le \textbf{pattern Abstract Factory} est un \textbf{design pattern créationnel} qui permet de créer des familles d'objets liés sans spécifier leur classe concrète.
          \item Il fournit une interface pour créer des objets liés ou dépendants dans une famille, tout en déléguant l'instanciation à des sous-classes concrètes.
          \item L'objectif est de garantir que les objets créés appartiennent à la même famille et sont compatibles entre eux.
          \item Ce pattern est utile lorsque le système doit travailler avec des objets liés, mais où le type exact des objets est déterminé dynamiquement au moment de l'exécution.
      \end{itemize}
    }{pic/diagrams/abstract-factory.png}
  \end{frame}

  \begin{frame}{Avantages du Pattern Abstract Factory}
    \begin{itemize}
        \item \textbf{Encapsulation des familles d'objets} : Fournit une interface pour créer des objets d'une même famille sans exposer leur instanciation.
        \item \textbf{Cohérence des produits} : Garantit que tous les objets créés dans une famille sont compatibles entre eux.
        \item \textbf{Respect du principe Open/Closed} : Permet d'ajouter de nouvelles familles de produits sans modifier le code existant.
        \item \textbf{Flexibilité} : Le pattern facilite le changement de famille de produits, car il suffit de changer la fabrique abstraite.
    \end{itemize}
  \end{frame}

  \begin{frame}{Inconvénients du Pattern Abstract Factory}
    \begin{itemize}
        \item \textbf{Complexité accrue} : Introduire des fabriques abstraites et des familles de produits peut augmenter la complexité globale du système.
        \item \textbf{Prolifération des classes} : Chaque nouvelle famille de produits et chaque produit au sein de cette famille nécessitent la création de nouvelles classes.
        \item \textbf{Difficulté d'extension des produits individuels} : Si vous devez ajouter un nouveau produit dans une famille existante, il peut être difficile d'étendre uniquement cette partie sans toucher à d'autres aspects.
    \end{itemize}
  \end{frame}

  \subsubsection{Différence avec Factory Method}

  \begin{frame}{Différence entre Abstract Factory et Factory Method}
    \begin{itemize}
        \item \textbf{Factory Method} :
        \begin{itemize}
            \item La Factory Method permet aux sous-classes de décider quel objet concret instancier.
            \item Il encapsule la logique d'instanciation dans une seule méthode spécifique et se concentre sur la création d'un seul produit à la fois.
        \end{itemize}
        \item \textbf{Abstract Factory} :
        \begin{itemize}
            \item L'Abstract Factory fournit une interface pour créer des familles d'objets reliés, souvent plusieurs produits en même temps.
            \item Le but est de garantir que tous les produits créés par une fabrique appartiennent à la même famille et sont compatibles entre eux.
        \end{itemize}
        \item \textbf{Différence clé} :
        \begin{itemize}
            \item Le pattern \textbf{Factory Method} est centré sur l'instanciation d'un objet unique, tandis que l'\textbf{Abstract Factory} crée des familles d'objets qui fonctionnent ensemble.
        \end{itemize}
    \end{itemize}
  \end{frame}

  \subsection{Singleton}
  \begin{frame}{Description du Pattern Singleton}
    \textimg{0.7}{
      \begin{itemize}
          \item Le \textbf{pattern Singleton} est un \textbf{design pattern créationnel} qui garantit qu'une classe n'a qu'une seule instance dans tout le programme.
          \item Il fournit un \textbf{point d'accès global} à cette instance unique.
          \item Le Singleton est souvent utilisé pour contrôler l'accès à des ressources partagées ou pour centraliser des services tels que des connexions à une base de données ou des fichiers de configuration.
      \end{itemize}
        
      }{pic/diagrams/singleton.png}
  \end{frame}

  \begin{frame}{Avantages du Pattern Singleton}
      \begin{itemize}
          \item \textbf{Contrôle de l'accès à une instance unique} : Assure qu'une seule instance est créée, centralisant l'accès aux ressources partagées.
          \item \textbf{Réduction de l'utilisation de la mémoire} : Une seule instance est créée, ce qui économise des ressources.
          \item \textbf{Facilité d'accès} : Un point d'accès global rend l'utilisation du Singleton simple dans tout le programme.
          \item \textbf{Contrôle centralisé} : Les modifications à l'instance unique sont disponibles pour tous les composants du système.
      \end{itemize}
  \end{frame}

  \begin{frame}{Inconvénients du Pattern Singleton}
      \begin{itemize}
          \item \textbf{Difficulté de test unitaire} : Difficile à tester à cause de la dépendance à l'instance globale.
          \item \textbf{Couplage global} : Introduit un couplage caché entre les différentes parties du programme, ce qui peut compliquer la maintenance.
          \item \textbf{Problèmes en multithreading} : Peut entraîner la création de plusieurs instances en environnement multithread sans synchronisation adéquate.
          \item \textbf{Difficulté d'évolution} : Si plusieurs instances sont requises à l'avenir, le code pourrait être difficile à adapter.
      \end{itemize}
  \end{frame}

  \subsection{Command}
  \begin{frame}
    \textbf{Contexte: }
    \par Imagine une télécommande universelle (le Remote Control). Cette télécommande peut être utilisée pour contrôler plusieurs appareils différents, 
    comme une lumière, un ventilateur, ou une télévision. Chaque appareil a des actions spécifiques (comme "allumer" et "éteindre"), 
    mais la télécommande ne sait pas comment chaque appareil fonctionne en interne. Elle se contente de déclencher des commandes associées aux boutons.
  \end{frame}
  \begin{frame}
    \only<1>{
    \textbf{Éléments clés du Command Pattern}
      Le \textbf{Command Pattern} est un patron de conception comportemental qui encapsule une requête en tant qu'objet. Cela permet de :
      \begin{itemize}
          \item Découpler l'émetteur d'une requête de son récepteur.
          \item Stocker, enregistrer et gérer les requêtes comme des objets indépendants.
      \end{itemize}
    }
    \only<2>{
      \textimg{0.6}{
        \begin{itemize}
          \item \textbf{Commande (Command)} : Interface définissant une méthode \texttt{execute()} pour exécuter la commande, avec option de \texttt{undo()}.
          \item \textbf{Commandes concrètes (Concrete Command)} : Implémentations spécifiques des commandes.
          \item \textbf{Récepteur (Receiver)} : L'objet sur lequel la commande agit.
          \item \textbf{Invocateur (Invoker)} : Celui qui déclenche la commande (ex: télécommande).
          \item \textbf{Client} : Crée les commandes et les configure avec les récepteurs.
        \end{itemize}
      }{pic/diagrams/command.png}
    }
  \end{frame}

  \begin{frame}{Avantages du Command Pattern}
    \begin{itemize}
        \item \textbf{Découplage} : Sépare l'émetteur (invocateur) et le récepteur (appareil).
        \item \textbf{Extensibilité} : Facile d'ajouter de nouvelles commandes sans modifier le code existant.
        \item \textbf{Support Undo/Redo} : Permet d'implémenter facilement des actions annulables.
        \item \textbf{Macro-Commandes} : Permet de composer des séquences d'actions complexes.
        \item \textbf{Enregistrement} : Possibilité d'enregistrer et de rejouer des commandes.
    \end{itemize}
  \end{frame}

  \begin{frame}{Inconvénients du Command Pattern}
    \begin{itemize}
        \item \textbf{Complexité accrue} : Introduction d'un grand nombre de classes, surtout si les commandes sont nombreuses.
        \item \textbf{Surcharge de code} : Pour chaque nouvelle action, une nouvelle commande doit être créée.
        \item \textbf{Complexité des annulations} : Gérer l'annulation d'une commande peut devenir difficile si les opérations sont complexes.
        \item \textbf{Usage non nécessaire} : Dans des scénarios simples, cela peut ajouter de la complexité inutilement.
    \end{itemize}
  \end{frame}

  \begin{frame}{Quand utiliser le Command Pattern ?}
    \begin{itemize}
        \item Lorsque vous avez besoin de \textbf{découpler} l'émission de la requête de son exécution.
        \item Si vous avez besoin de \textbf{fonctionnalités Undo/Redo}.
        \item Si vous devez \textbf{mettre en file d'attente} des commandes pour une exécution ultérieure.
        \item Lorsque vous voulez créer des \textbf{séquences complexes} d'actions.
    \end{itemize}
    \pause
    Toutefois, si le système est simple ou n'a pas besoin de cette flexibilité, le Command Pattern peut ajouter une complexité inutile.
  \end{frame}

  \subsection{Adapter}
  \begin{frame}{Pattern Adapter}
    \only<1>{
      Le \textbf{Pattern Adapter} est un patron de conception structurel qui permet de rendre deux interfaces incompatibles compatibles. 
      Il agit comme un pont entre une classe existante et une nouvelle interface.
      \begin{itemize}
          \item Utilisé pour réutiliser du code existant avec une interface différente.
          \item Permet l'interopérabilité avec des bibliothèques externes ayant des interfaces incompatibles.
      \end{itemize}
    }
    \only<2>{
      \textimg{0.6}{
        \textbf{{Éléments clés du Pattern Adapter}}
        \begin{itemize}
            \item \textbf{Client} : Utilise l'interface cible.
            \item \textbf{Target Interface} : L'interface attendue par le client.
            \item \textbf{Adaptee} : L'ancienne classe que l'on souhaite adapter.
            \item \textbf{Adapter} : Traduit l'interface de l'Adaptee pour qu'elle soit compatible avec la Target Interface.
        \end{itemize}
      }{pic/diagrams/adapter.png}
    }
  \end{frame}

  \begin{frame}{Avantages du Pattern Adapter}
    \begin{itemize}
        \item \textbf{Réutilisation du code} : Utilise des classes existantes avec des interfaces différentes.
        \item \textbf{Découplage} : Sépare les interfaces tout en permettant leur compatibilité.
        \item \textbf{Flexibilité} : Permet d'intégrer des API ou bibliothèques externes sans modification.
    \end{itemize}
  \end{frame}


  \begin{frame}{Inconvénients du Pattern Adapter}
    \begin{itemize}
        \item \textbf{Complexité supplémentaire} : Introduction d'une nouvelle couche de code.
        \item \textbf{Multiplication des classes} : Augmente le nombre de classes à maintenir.
    \end{itemize}
  \end{frame}

  % Slide 6: Quand utiliser le Pattern Adapter ?
  \begin{frame}{Quand utiliser le Pattern Adapter ?}
    \begin{itemize}
        \item Lorsqu'il est nécessaire de rendre des classes existantes compatibles avec une interface différente.
        \item Lorsque des bibliothèques ou des API externes ont une interface incompatible avec votre système.
    \end{itemize}
  \end{frame}

  \

  \subsection{Façade}
  \begin{frame}{Pattern Façade}
    Le \textbf{Pattern Façade} est un patron de conception structurel qui fournit une interface simplifiée pour interagir avec un ou plusieurs sous-systèmes complexes. 
    \begin{itemize}
        \item Masque la complexité interne d'un système.
        \item Fournit un point d'entrée unique pour accéder aux sous-systèmes.
    \end{itemize}
  \end{frame}

  \begin{frame}{Éléments clés du Pattern Façade}
    \textimg{0.6}{
      \begin{itemize}
          \item \textbf{Façade} : Classe offrant une interface simplifiée pour interagir avec les sous-systèmes.
          \item \textbf{Sous-systèmes} : Les composants ou modules qui réalisent les actions réelles.
          \item \textbf{Client} : Utilise la façade pour accéder aux fonctionnalités des sous-systèmes sans connaître leur complexité.
      \end{itemize}
    }{pic/diagrams/facade.png}
  \end{frame}

  \begin{frame}{Avantages du Pattern Façade}
    \begin{itemize}
        \item \textbf{Simplification} : Cache la complexité des sous-systèmes, fournissant une interface simple.
        \item \textbf{Découplage} : Sépare le client des sous-systèmes internes, rendant le système plus flexible.
        \item \textbf{Code plus lisible} : Réduit les appels multiples, simplifiant le code client.
        \item \textbf{Facilité d'intégration} : La façade absorbe les changements des sous-systèmes, sans affecter les utilisateurs.
    \end{itemize}
  \end{frame}

  \begin{frame}{Inconvénients du Pattern Façade}
    \begin{itemize}
        \item \textbf{Masquage des fonctionnalités avancées} : Certaines fonctionnalités des sous-systèmes peuvent être inaccessibles via la façade.
        \item \textbf{Responsabilité accrue} : La façade peut devenir trop volumineuse si elle gère trop de logique interne.
        \item \textbf{Complexité cachée} : La façade masque la complexité sous-jacente, ce qui peut poser problème en cas de débogage.
    \end{itemize}
  \end{frame}

  \begin{frame}{Quand utiliser le Pattern Façade ?}
    \begin{itemize}
        \item Lorsque le système est complexe et que l'on souhaite fournir une interface simplifiée.
        \item Pour réduire le couplage entre les sous-systèmes et le client.
        \item Pour intégrer facilement une bibliothèque ou un ensemble d'API complexes en exposant une interface plus intuitive.
    \end{itemize}
  \end{frame}

  \subsection{Template method}
  \begin{frame}{Pattern Template Method}
    Le \textbf{Template Method Pattern} est un patron de conception comportemental qui permet de définir la structure d'un algorithme dans une méthode 
    tout en laissant certaines étapes spécifiques aux sous-classes.
    
    \begin{itemize}
        \item La classe parent définit le squelette de l'algorithme.
        \item Les sous-classes peuvent redéfinir certaines étapes sans modifier l'algorithme global.
    \end{itemize}
  \end{frame}


  \begin{frame}{Structure du Template Method}
    \textimg{0.6}{
      \begin{itemize}
          \item \textbf{Classe abstraite} : Définit la méthode template et certaines étapes par défaut.
          \item \textbf{Méthode Template} : La méthode qui définit l'ordre des étapes de l'algorithme.
          \item \textbf{Sous-classes} : Implémentent ou redéfinissent certaines étapes spécifiques.
      \end{itemize}
    }{pic/diagrams/template-method.png}
  \end{frame}


  \begin{frame}{Avantages du Template Method Pattern}
    \begin{itemize}
        \item \textbf{Réutilisation du code} : Le squelette de l'algorithme est défini une seule fois dans la classe parent.
        \item \textbf{Contrôle centralisé} : L'algorithme est défini au même endroit, ce qui facilite la maintenance.
        \item \textbf{Personnalisation} : Les sous-classes peuvent modifier certaines étapes sans changer la structure globale.
        \item \textbf{Inversion de contrôle} : Les sous-classes décident de certains détails, tandis que l'ordre global est géré par la classe parent.
    \end{itemize}
  \end{frame}

  \begin{frame}{Inconvénients du Template Method Pattern}
    \begin{itemize}
        \item \textbf{Rigidité} : L'algorithme global est figé dans la classe parent, limitant la flexibilité des sous-classes.
        \item \textbf{Dépendance} : Les sous-classes sont fortement couplées à la structure de la classe parent.
        \item \textbf{Difficulté à comprendre} : La combinaison de la méthode template et des méthodes redéfinies peut rendre la structure plus difficile à suivre.
    \end{itemize}
  \end{frame}

  \begin{frame}{Quand utiliser le Template Method Pattern ?}
    \begin{itemize}
        \item Lorsque plusieurs classes partagent un algorithme commun, mais que certaines étapes doivent être personnalisées.
        \item Lorsque vous souhaitez centraliser et réutiliser une structure commune tout en permettant des variations dans les détails.
        \item Pour éviter la duplication de code dans les sous-classes qui partagent la même structure générale.
    \end{itemize}
  \end{frame}

  \subsection{iterator}
  \begin{frame}{Pattern Iterator}
    Le \textbf{Pattern Iterator} est un patron de conception comportemental qui permet de parcourir les éléments d'une collection sans exposer sa structure interne.
    
    \begin{itemize}
        \item Fournit une interface uniforme pour accéder aux éléments.
        \item Permet d'itérer sur différentes collections de manière indépendante de leur implémentation.
    \end{itemize}
  \end{frame}

  % Slide 2: Structure du Pattern Iterator
  \begin{frame}{Structure du Pattern Iterator}
    \textimg{0.6}{
      \begin{itemize}
          \item \textbf{Itérateur (Iterator)} : Définit les méthodes pour parcourir les éléments (e.g. \texttt{hasNext()}, \texttt{next()}).
          \item \textbf{Collection (Aggregate)} : Fournit une méthode pour obtenir un itérateur.
          \item \textbf{Itérateur concret (Concrete Iterator)} : Implémente l'interface Iterator pour une collection spécifique.
          \item \textbf{Collection concrète (Concrete Aggregate)} : Implémente la structure des données et renvoie l'itérateur correspondant.
      \end{itemize}
    }{pic/diagrams/iterator.png}
  \end{frame}


  \begin{frame}{Avantages du Pattern Iterator}
    \begin{itemize}
        \item \textbf{Encapsulation} : Masque la structure interne des collections.
        \item \textbf{Uniformité} : Offre une interface uniforme pour parcourir différentes collections.
        \item \textbf{Séparation des préoccupations} : La logique de parcours est déplacée dans un itérateur, pas dans la collection elle-même.
        \item \textbf{Simplification} : Rend le code client plus simple et plus propre, en le séparant des détails de parcours.
    \end{itemize}
  \end{frame}

  \begin{frame}{Inconvénients du Pattern Iterator}
    \begin{itemize}
        \item \textbf{Complexité supplémentaire} : Introduit une couche supplémentaire, surtout pour des collections simples.
        \item \textbf{Accès unidirectionnel} : La plupart des itérateurs ne permettent qu'un parcours dans un seul sens.
        \item \textbf{Multiplication des classes} : Chaque collection nécessite souvent son propre itérateur concret.
        \item \textbf{Performance} : Peut avoir une légère surcharge par rapport à l'accès direct à la collection.
    \end{itemize}
  \end{frame}

  \begin{frame}{Quand utiliser le Pattern Iterator ?}
    \begin{itemize}
        \item Lorsque vous devez parcourir une collection sans exposer sa structure interne.
        \item Lorsque vous avez différentes collections (listes, tableaux, ensembles) que vous souhaitez traiter de manière uniforme.
        \item Pour simplifier le code client et séparer les responsabilités.
        \item Si vous voulez éviter de dupliquer la logique de parcours dans chaque classe client.
    \end{itemize}
  \end{frame}

  \subsection{State}
  \begin{frame}{Pattern State}
    \only<1>{
      Le \textbf{Pattern State} est un patron de conception comportemental qui permet à un objet de changer son comportement lorsque son état interne change.
      \begin{itemize}
          \item Encapsule chaque état dans une classe distincte.
          \item Permet de changer dynamiquement de comportement sans modifier le code client.
      \end{itemize}
    }
    \only<2>{
      \textimg{0.6}{
        \begin{itemize}
            \item \textbf{Contexte (Context)} : L'objet principal qui contient une référence à son état actuel.
            \item \textbf{État (State)} : Interface ou classe abstraite déclarant des méthodes spécifiques à chaque état.
            \item \textbf{États concrets (Concrete State)} : Classes qui implémentent l'interface `State` et définissent le comportement pour chaque état.
        \end{itemize}
      }{pic/diagrams/state.png}
    }
  \end{frame}

  \begin{frame}{Avantages du Pattern State}
    \begin{itemize}
        \item \textbf{Élimination des structures conditionnelles} : Pas besoin de longs `if` ou `switch`.
        \item \textbf{Responsabilité unique} : Chaque état a sa propre classe, responsable de son comportement.
        \item \textbf{Extension facile} : Ajouter de nouveaux états est simple et ne nécessite pas de modification du code existant.
        \item \textbf{Changement dynamique d'état} : Le comportement peut être modifié à la volée sans affecter le client.
    \end{itemize}
  \end{frame}

  \begin{frame}{Inconvénients du Pattern State}
    \begin{itemize}
        \item \textbf{Multiplication des classes} : Crée plusieurs classes pour chaque état, ce qui peut devenir complexe.
        \item \textbf{Changement d'état implicite} : Peut être difficile à suivre si les transitions ne sont pas bien documentées.
        \item \textbf{Complexité pour systèmes simples} : Peut ajouter de la complexité inutile dans des systèmes simples.
    \end{itemize}
  \end{frame}


  \begin{frame}{Quand utiliser le Pattern State ?}
    \begin{itemize}
        \item Lorsque le comportement d'un objet dépend de son état interne.
        \item Lorsque vous voulez éviter des structures conditionnelles complexes pour gérer les états.
        \item Si vous avez plusieurs états avec des comportements distincts et souhaitez encapsuler ces comportements.
    \end{itemize}
  \end{frame}

  \section{Principes de Conception (SOLID)}

  \begin{frame}{Programmer pour une interface plutôt que pour une implémentation}
  \begin{itemize}
      \item Coder contre des abstractions (interfaces ou classes abstraites) pour rendre le code plus flexible.
      \item Exemple : Utiliser une interface \texttt{Animal} plutôt que directement \texttt{Chien} ou \texttt{Chat}.
  \end{itemize}
  \end{frame}

  \begin{frame}{Favoriser la composition sur l'héritage}
  \begin{itemize}
      \item Utiliser la composition pour combiner des objets plutôt que l'héritage pour réutiliser du code.
      \item Exemple : Une classe \texttt{Voiture} peut avoir un objet \texttt{Moteur} au lieu d'hériter d'une classe \texttt{Moteur}.
  \end{itemize}
  \end{frame}

  \begin{frame}{Principe Ouvert/Fermé (Open/Closed Principle - OCP)}
  \begin{itemize}
      \item Les classes doivent être ouvertes à l'extension mais fermées à la modification.
      \item Exemple : Ajouter de nouveaux types de paiements via une interface \texttt{ModePaiement} sans changer la classe principale.
  \end{itemize}
  \end{frame}

  \begin{frame}{Responsabilité Unique (Single Responsibility Principle - SRP)}
  \begin{itemize}
      \item Une classe doit avoir une seule responsabilité, donc une seule raison de changer.
      \item Exemple : Séparer la facturation et l'envoi d'e-mails dans deux classes distinctes.
  \end{itemize}
  \end{frame}

  \begin{frame}{Substitution de Liskov (Liskov Substitution Principle - LSP)}
  \begin{itemize}
      \item Les sous-classes doivent pouvoir remplacer les super-classes sans modifier le comportement du programme.
      \item Exemple : Si \texttt{Oiseau} a une méthode \texttt{voler()}, \texttt{Pingouin} ne doit pas hériter directement de \texttt{Oiseau} s'il ne vole pas.
  \end{itemize}
  \end{frame}

  \begin{frame}{Ségrégation des Interfaces (Interface Segregation Principle - ISP)}
  \begin{itemize}
      \item Mieux vaut plusieurs interfaces spécifiques qu'une seule interface générale.
      \item Exemple : Avoir des interfaces comme \texttt{Volant}, \texttt{Nageant}, plutôt qu'une unique interface \texttt{Animal}.
  \end{itemize}
  \end{frame}

  \begin{frame}{Inversion des Dépendances (Dependency Inversion Principle - DIP)}
  \begin{itemize}
      \item Les modules de haut niveau ne doivent pas dépendre des modules de bas niveau. Tous deux doivent dépendre d'abstractions.
      \item Exemple : \texttt{GestionUtilisateur} doit dépendre de \texttt{ServiceNotification}, pas de \texttt{ServiceEmail}.
  \end{itemize}
  \end{frame}

  \begin{frame}{Encapsuler ce qui varie}
  \begin{itemize}
      \item Identifier et isoler les parties du code qui changent fréquemment pour minimiser l'impact des changements.
      \item Exemple : Encapsuler les comportements comme \texttt{marcher} ou \texttt{attaquer} dans des classes séparées dans un jeu vidéo.
  \end{itemize}
  \end{frame}

  \begin{frame}{Favoriser le couplage faible}
  \begin{itemize}
      \item Réduire les dépendances directes entre objets pour une meilleure modularité et indépendance.
      \item Exemple : Utiliser des événements ou des interfaces pour éviter les relations directes entre objets.
  \end{itemize}
  \end{frame}

  \begin{frame}{Parler uniquement avec vos amis (Law of Demeter)}
  \begin{itemize}
      \item Limiter les interactions directes d'un objet uniquement aux objets avec lesquels il est directement en relation.
      \item Exemple : Une classe \texttt{Commande} ne devrait interagir qu'avec \texttt{Client}, pas avec \texttt{Adresse}.
  \end{itemize}
  \end{frame}

  \begin{frame}{Dépendre d'une abstraction, pas d'une implémentation}
  \begin{itemize}
      \item Toujours dépendre d'abstractions (interfaces ou classes abstraites) et non d'implémentations concrètes.
      \item Exemple : \texttt{PaiementService} devrait dépendre de \texttt{Transaction} et non de \texttt{TransactionBancaire}.
  \end{itemize}
  \end{frame}

  \begin{frame}{Don't call us, we'll call you (Inversion de contrôle)}
  \begin{itemize}
      \item Les objets doivent attendre qu'un système plus haut niveau les appelle, plutôt que d'initier eux-mêmes des actions.
      \item Exemple : Les événements dans une interface utilisateur : le système appelle l'action en réponse à un événement.
  \end{itemize}
  \end{frame}

\end{document}